\documentclass[]{article}
\usepackage{lmodern}
\usepackage{amssymb,amsmath}
\usepackage{ifxetex,ifluatex}
\usepackage{fixltx2e} % provides \textsubscript
\ifnum 0\ifxetex 1\fi\ifluatex 1\fi=0 % if pdftex
  \usepackage[T1]{fontenc}
  \usepackage[utf8]{inputenc}
\else % if luatex or xelatex
  \ifxetex
    \usepackage{mathspec}
  \else
    \usepackage{fontspec}
  \fi
  \defaultfontfeatures{Ligatures=TeX,Scale=MatchLowercase}
\fi
% use upquote if available, for straight quotes in verbatim environments
\IfFileExists{upquote.sty}{\usepackage{upquote}}{}
% use microtype if available
\IfFileExists{microtype.sty}{%
\usepackage{microtype}
\UseMicrotypeSet[protrusion]{basicmath} % disable protrusion for tt fonts
}{}
\usepackage[margin=1in]{geometry}
\usepackage{hyperref}
\hypersetup{unicode=true,
            pdftitle={Predicción de primas de seguro a través de los modelos de Poisson compuestos de Tweedie reforzados con árbol de gradiente},
            pdfborder={0 0 0},
            breaklinks=true}
\urlstyle{same}  % don't use monospace font for urls
\usepackage{graphicx,grffile}
\makeatletter
\def\maxwidth{\ifdim\Gin@nat@width>\linewidth\linewidth\else\Gin@nat@width\fi}
\def\maxheight{\ifdim\Gin@nat@height>\textheight\textheight\else\Gin@nat@height\fi}
\makeatother
% Scale images if necessary, so that they will not overflow the page
% margins by default, and it is still possible to overwrite the defaults
% using explicit options in \includegraphics[width, height, ...]{}
\setkeys{Gin}{width=\maxwidth,height=\maxheight,keepaspectratio}
\IfFileExists{parskip.sty}{%
\usepackage{parskip}
}{% else
\setlength{\parindent}{0pt}
\setlength{\parskip}{6pt plus 2pt minus 1pt}
}
\setlength{\emergencystretch}{3em}  % prevent overfull lines
\providecommand{\tightlist}{%
  \setlength{\itemsep}{0pt}\setlength{\parskip}{0pt}}
\setcounter{secnumdepth}{0}
% Redefines (sub)paragraphs to behave more like sections
\ifx\paragraph\undefined\else
\let\oldparagraph\paragraph
\renewcommand{\paragraph}[1]{\oldparagraph{#1}\mbox{}}
\fi
\ifx\subparagraph\undefined\else
\let\oldsubparagraph\subparagraph
\renewcommand{\subparagraph}[1]{\oldsubparagraph{#1}\mbox{}}
\fi

%%% Use protect on footnotes to avoid problems with footnotes in titles
\let\rmarkdownfootnote\footnote%
\def\footnote{\protect\rmarkdownfootnote}

%%% Change title format to be more compact
\usepackage{titling}

% Create subtitle command for use in maketitle
\newcommand{\subtitle}[1]{
  \posttitle{
    \begin{center}\large#1\end{center}
    }
}

\setlength{\droptitle}{-2em}

  \title{Predicción de primas de seguro a través de los modelos de Poisson
compuestos de Tweedie reforzados con árbol de gradiente}
    \pretitle{\vspace{\droptitle}\centering\huge}
  \posttitle{\par}
    \author{}
    \preauthor{}\postauthor{}
    \date{}
    \predate{}\postdate{}
  

\begin{document}
\maketitle

\subsection{Resumen}\label{resumen}

Tweedie GLM es un método ampliamente utilizado para predecir primas de
seguros. Sin embargo, la estructura de la media logarítmica está
restringida a una forma lineal en el GLM de Tweedie, que puede ser
demasiado rígida para muchas aplicaciones. Como mejor alternativa,
proponemos un algoritmo de aumento de árbol de gradiente y lo aplicamos
a los modelos de Poisson de Tweedie compuestos para primas puras.
Utilizamos un enfoque de probabilidad de perfil para estimar los
parámetros de índice y dispersión. Nuestro método es capaz de ajustar un
modelo flexible no lineal de Tweedie y capturar interacciones complejas
entre predictores. Un estudio de simulación confirma el excelente
rendimiento de predicción de nuestro método. Como una aplicación,
aplicamos nuestro método a los datos de reclamos de seguros de
automóviles y mostramos que el nuevo método es superior a los métodos
existentes en el sentido de que genera predicciones de primas más
precisas. Ayudando así a resolver el problema de selección adversa.
Hemos implementado nuestro método en un paquete R fácil de usar que
también incluye una buena herramienta de visualización para interpretar
el modelo ajustado.

\subsection{Introducción}\label{introduccion}

Uno de los problemas más importantes en el negocio de seguros es
establecer la prima para los clientes (asegurados). En un mercado
competitivo, es ventajoso para la aseguradora cobrar una prima justa de
acuerdo con la pérdida esperada del asegurado. En el seguro de automóvil
personal, por ejemplo, si una compañía de seguros cobra demasiado por
los conductores antiguos y los de los conductores jóvenes, los
conductores antiguos cambiarán a sus competidores, y las pólizas
restantes para los conductores jóvenes tendrán un precio bajo. Esto da
como resultado el problema de selección adversa (Dionne et al., 2001):
la aseguradora pierde políticas rentables y se queda con malos riesgos,
lo que resulta en pérdidas económicas en ambos sentidos. Para establecer
adecuadamente las primas para los clientes de la aseguradora, una tarea
crucial es predecir el tamaño de las reclamaciones reales (actualmente
imprevisibles). En este documento, nos centraremos en modelar la pérdida
de reclamaciones, aunque otros ingredientes como la carga de seguridad,
los costos administrativos, el costo de capital y las ganancias también
son factores importantes para establecer la prima. Una dificultad para
modelar las afirmaciones es que la distribución suele ser muy sesgada a
la derecha, mezclada con una masa puntual en cero. Este tipo de datos no
se puede transformar en normalidad mediante la transformación del poder
y, a menudo, se requiere un tratamiento especial en las reclamaciones
cero. Como ejemplo, la Figura 1 muestra el histograma de los datos de un
reclamo de seguro de automóvil (Yip y Yau, 2005), en el que hay 6, 290
registros de pólizas con cero reclamaciones y 4,006 registros de pólizas
con pérdidas positivas. La necesidad de modelos predictivos surge del
hecho de que la pérdida esperada depende en gran medida de las
características de una póliza individual, como la antigüedad y los
puntos de registro del vehículo motorizado del asegurado, la densidad de
población del área residencial del asegurado y la edad y el modelo del
vehículo. . Los métodos tradicionales utilizaron modelos lineales
generalizados (GLM; Nelder y Wedderburn, 1972) para modelar el tamaño de
la reclamación (por ejemplo, Renshaw, 1994; Haberman y Renshaw, 1996).
Sin embargo, los autores de los artículos anteriores realizaron sus
análisis en un subconjunto de las políticas, que tienen al menos una
reclamación. Los enfoques alternativos han empleado los modelos de Tobit
al tratar resultados cero como censurados debajo de algunos puntos de
corte (Van de Ven y van Praag, 1981; Duchas 006 registros de pólizas con
pérdidas positivas. La necesidad de modelos predictivos surge del hecho
de que la pérdida esperada depende en gran medida de las características
de una póliza individual, como la antigüedad y los puntos de registro
del vehículo motorizado del asegurado, la densidad de población del área
residencial del asegurado y la edad y el modelo del vehículo. . Los
métodos tradicionales utilizaron modelos lineales generalizados (GLM;
Nelder y Wedderburn, 1972) para modelar el tamaño de la reclamación (por
ejemplo, Renshaw, 1994; Haberman y Renshaw, 1996). Sin embargo, los
autores de los artículos anteriores realizaron sus análisis en un
subconjunto de las políticas, que tienen al menos una reclamación. Los
enfoques alternativos han empleado los modelos de Tobit al tratar
resultados cero como censurados debajo de algunos puntos de corte (Van
de Ven y van Praag, 1981; Duchas 006 registros de pólizas con pérdidas
positivas. La necesidad de modelos predictivos surge del hecho de que la
pérdida esperada depende en gran medida de las características de una
póliza individual, como la antigüedad y los puntos de registro del
vehículo motorizado del asegurado, la densidad de población del área
residencial del asegurado y la edad y el modelo del vehículo. . Los
métodos tradicionales utilizaron modelos lineales generalizados (GLM;
Nelder y Wedderburn, 1972) para modelar el tamaño de la reclamación (por
ejemplo, Renshaw, 1994; Haberman y Renshaw, 1996). Sin embargo, los
autores de los artículos anteriores realizaron sus análisis en un
subconjunto de las políticas, que tienen al menos una reclamación. Los
enfoques alternativos han empleado los modelos de Tobit al tratar
resultados cero como censurados debajo de algunos puntos de corte (Van
de Ven y van Praag, 1981; Duchas La necesidad de modelos predictivos
surge del hecho de que la pérdida esperada depende en gran medida de las
características de una póliza individual, como la antigüedad y los
puntos de registro del vehículo motorizado del asegurado, la densidad de
población del área residencial del asegurado y la edad y el modelo del
vehículo. . Los métodos tradicionales utilizaron modelos lineales
generalizados (GLM; Nelder y Wedderburn, 1972) para modelar el tamaño de
la reclamación (por ejemplo, Renshaw, 1994; Haberman y Renshaw, 1996).
Sin embargo, los autores de los artículos anteriores realizaron sus
análisis en un subconjunto de las políticas, que tienen al menos una
reclamación. Los enfoques alternativos han empleado los modelos de Tobit
al tratar resultados cero como censurados debajo de algunos puntos de
corte (Van de Ven y van Praag, 1981; Duchas La necesidad de modelos
predictivos surge del hecho de que la pérdida esperada depende en gran
medida de las características de una póliza individual, como la
antigüedad y los puntos de registro del vehículo motorizado del
asegurado, la densidad de población del área residencial del asegurado y
la edad y el modelo del vehículo. . Los métodos tradicionales utilizaron
modelos lineales generalizados (GLM; Nelder y Wedderburn, 1972) para
modelar el tamaño de la reclamación (por ejemplo, Renshaw, 1994;
Haberman y Renshaw, 1996). Sin embargo, los autores de los artículos
anteriores realizaron sus análisis en un subconjunto de las políticas,
que tienen al menos una reclamación. Los enfoques alternativos han
empleado los modelos de Tobit al tratar resultados cero como censurados
debajo de algunos puntos de corte (Van de Ven y van Praag, 1981; Duchas
Los métodos tradicionales utilizaron modelos lineales generalizados
(GLM; Nelder y Wedderburn, 1972) para modelar el tamaño de la
reclamación (por ejemplo, Renshaw, 1994; Haberman y Renshaw, 1996). Sin
embargo, los autores de los artículos anteriores realizaron sus análisis
en un subconjunto de las políticas, que tienen al menos una reclamación.
Los enfoques alternativos han empleado los modelos de Tobit al tratar
resultados cero como censurados debajo de algunos puntos de corte (Van
de Ven y van Praag, 1981; Duchas Los métodos tradicionales utilizaron
modelos lineales generalizados (GLM; Nelder y Wedderburn, 1972) para
modelar el tamaño de la reclamación (por ejemplo, Renshaw, 1994;
Haberman y Renshaw, 1996). Sin embargo, los autores de los artículos
anteriores realizaron sus análisis en un subconjunto de las políticas,
que tienen al menos una reclamación. Los enfoques alternativos han
empleado los modelos de Tobit al tratar resultados cero como censurados
debajo de algunos puntos de corte (Van de Ven y van Praag, 1981; Duchas

2

 y Shotick, 1994), pero estos enfoques se basan en un supuesto de
normalidad de la respuesta latente. Alternativamente, Jørgensen y de
Souza (1994) y Smyth y Jørgensen (2002) utilizaron GLM con un resultado
distribuido de Tweedie para modelar simultáneamente la frecuencia y la
severidad de las reclamaciones de seguros. Asumen la llegada de reclamos
de Poisson y el monto distribuido gamma para reclamos individuales, de
modo que el tamaño del monto total del reclamo sigue una distribución de
Poisson compuesta de Tweedie. Debido a su capacidad para modelar
simultáneamente los ceros y los resultados positivos continuos, Tweedie
GLM ha sido un método ampliamente utilizado en estudios actuariales
(Mildenhall, 1999; Murphy et al., 2000; Peters Descargado por {[}McGill
University Library{]} a las 18:16 28 de junio de 2016

et al., 2008). A pesar de la popularidad de Tweedie GLM, una limitación
importante es que la estructura de la media logarítmica está restringida
a una forma lineal, que puede ser demasiado rígida para aplicaciones
reales. En el seguro de automóviles, por ejemplo, se sabe que el riesgo
no disminuye de manera monotónica a medida que aumenta la edad (Anstey
et al., 2005). Aunque la no linealidad se puede modelar agregando
splines (Zhang, 2011), las splines de bajo grado a menudo son
inadecuadas para capturar la no linealidad en los datos, mientras que
las splines de alto grado a menudo resultan en el problema de ajuste
excesivo que produce estimaciones inestables. Los modelos de aditivos
generalizados (GAM; Hastie y Tibshirani, 1990; Wood, 2006) superan la
suposición lineal restrictiva de los GLM y pueden modelar las variables
continuas mediante funciones suaves estimadas a partir de los datos. Sin
embargo, la estructura del modelo debe determinarse a priori. Es decir,
uno tiene que especificar los efectos principales y los efectos de
interacción que se utilizarán en el modelo. Como resultado, es probable
que la mala especificación de los efectos no ignorables afecte
negativamente la precisión de la predicción. En este documento, nuestro
objetivo es modelar el tamaño de la reclamación de seguros mediante un
modelo de Poisson compuesto no paramétrico de Tweedie, y proponer un
algoritmo de aumento de árbol de gradiente (TDboost en adelante) para
que se ajuste a este modelo. También implementamos el método propuesto
como un paquete R fácil de usar, que está disponible públicamente. El
aumento de gradiente es uno de los algoritmos de aprendizaje automático
más exitosos para la regresión y clasificación no paramétrica. Boosting
adapta una gran cantidad de modelos de predicción relativamente simples
llamados aprendices base en un aprendiz conjunto para lograr un alto
rendimiento de predicción. uno tiene que especificar los efectos
principales y los efectos de interacción que se utilizarán en el modelo.
Como resultado, es probable que la mala especificación de los efectos no
ignorables afecte negativamente la precisión de la predicción. En este
documento, nuestro objetivo es modelar el tamaño de la reclamación de
seguros mediante un modelo de Poisson compuesto no paramétrico de
Tweedie, y proponer un algoritmo de aumento de árbol de gradiente
(TDboost en adelante) para que se ajuste a este modelo. También
implementamos el método propuesto como un paquete R fácil de usar, que
está disponible públicamente. El aumento de gradiente es uno de los
algoritmos de aprendizaje automático más exitosos para la regresión y
clasificación no paramétrica. Boosting adapta una gran cantidad de
modelos de predicción relativamente simples llamados aprendices base en
un aprendiz conjunto para lograr un alto rendimiento de predicción. uno
tiene que especificar los efectos principales y los efectos de
interacción que se utilizarán en el modelo. Como resultado, es probable
que la mala especificación de los efectos no ignorables afecte
negativamente la precisión de la predicción. En este documento, nuestro
objetivo es modelar el tamaño de la reclamación de seguros mediante un
modelo de Poisson compuesto no paramétrico de Tweedie, y proponer un
algoritmo de aumento de árbol de gradiente (TDboost en adelante) para
que se ajuste a este modelo. También implementamos el método propuesto
como un paquete R fácil de usar, que está disponible públicamente. El
aumento de gradiente es uno de los algoritmos de aprendizaje automático
más exitosos para la regresión y clasificación no paramétrica. Boosting
adapta una gran cantidad de modelos de predicción relativamente simples
llamados aprendices base en un aprendiz conjunto para lograr un alto
rendimiento de predicción. La especificación errónea de efectos no
ignorables puede afectar negativamente la precisión de la predicción. En
este documento, nuestro objetivo es modelar el tamaño de la reclamación
de seguros mediante un modelo de Poisson compuesto no paramétrico de
Tweedie, y proponer un algoritmo de aumento de árbol de gradiente
(TDboost en adelante) para que se ajuste a este modelo. También
implementamos el método propuesto como un paquete R fácil de usar, que
está disponible públicamente. El aumento de gradiente es uno de los
algoritmos de aprendizaje automático más exitosos para la regresión y
clasificación no paramétrica. Boosting adapta una gran cantidad de
modelos de predicción relativamente simples llamados aprendices base en
un aprendiz conjunto para lograr un alto rendimiento de predicción. La
especificación errónea de efectos no ignorables puede afectar
negativamente la precisión de la predicción. En este documento, nuestro
objetivo es modelar el tamaño de la reclamación de seguros mediante un
modelo de Poisson compuesto no paramétrico de Tweedie, y proponer un
algoritmo de aumento de árbol de gradiente (TDboost en adelante) para
que se ajuste a este modelo. También implementamos el método propuesto
como un paquete R fácil de usar, que está disponible públicamente. El
aumento de gradiente es uno de los algoritmos de aprendizaje automático
más exitosos para la regresión y clasificación no paramétrica. Boosting
adapta una gran cantidad de modelos de predicción relativamente simples
llamados aprendices base en un aprendiz conjunto para lograr un alto
rendimiento de predicción. y proponer un algoritmo de aumento de árbol
de gradiente (TDboost a partir de ahora) para ajustar este modelo.
También implementamos el método propuesto como un paquete R fácil de
usar, que está disponible públicamente. El aumento de gradiente es uno
de los algoritmos de aprendizaje automático más exitosos para la
regresión y clasificación no paramétrica. Boosting adapta una gran
cantidad de modelos de predicción relativamente simples llamados
aprendices base en un aprendiz conjunto para lograr un alto rendimiento
de predicción. y proponer un algoritmo de aumento de árbol de gradiente
(TDboost a partir de ahora) para ajustar este modelo. También
implementamos el método propuesto como un paquete R fácil de usar, que
está disponible públicamente. El aumento de gradiente es uno de los
algoritmos de aprendizaje automático más exitosos para la regresión y
clasificación no paramétrica. Boosting adapta una gran cantidad de
modelos de predicción relativamente simples llamados aprendices base en
un aprendiz conjunto para lograr un alto rendimiento de predicción.

3

 mance El trabajo seminal sobre el algoritmo de impulso llamado
AdaBoost (Freund y Schapire, 1997) se propuso originalmente para
problemas de clasificación. Más tarde, Breiman (1998) y Breiman (1999)
señalaron una conexión importante entre el algoritmo de AdaBoost y un
algoritmo de descenso de gradiente funcional. Friedman et al. (2000) y
Hastie et al. (2009) desarrolló una visión estadística de los métodos de
refuerzo y gradiente propuesto para la clasificación y la regresión. Hay
una gran cantidad de literatura sobre el impulso. Referimos a los
lectores interesados a B¨ hlmann y Hothorn (2007) u para una revisión
exhaustiva de los algoritmos de impulso. Descargado por {[}McGill
University Library{]} a las 18:16 28 de junio de 2016

El modelo TDboost está motivado por el éxito comprobado de impulsar el
aprendizaje automático para problemas de clasificación y regresión
(Friedman, 2001, 2002; Hastie et al., 2009). Sus ventajas son triples.
Primero, la estructura del modelo de TDboost se aprende de los datos y
no se predetermina, evitando así una especificación explícita del
modelo. Las no linealidades, discontinuidades, interacciones complejas y
de orden superior se incorporan naturalmente al modelo para reducir el
sesgo de modelado potencial y para producir un alto rendimiento
predictivo, lo que permite a TDboost servir como un modelo de referencia
en la calificación de pólizas de seguro, orientar las prácticas de
fijación de precios y facilitar Esfuerzos de mercadeo. La selección de
características se realiza como parte integral del procedimiento.
Además, TDboost maneja las variables de predictor y respuesta de
cualquier tipo sin la necesidad de transformación, y es muy robusto a
los valores atípicos. Los valores faltantes en los predictores se
manejan casi sin pérdida de información (Elith et al., 2008). Todas
estas propiedades hacen de TDboost una herramienta más atractiva para el
modelado de primas de seguros. Por otro lado, reconocemos que sus
resultados no son tan sencillos como los del modelo Tweedie GLM. Sin
embargo, TDboost no tiene que ser considerado como una caja negra. Puede
proporcionar resultados interpretables, por medio de las parcelas de
dependencia parcial, y la importancia relativa de los predictores. El
resto de este documento está organizado de la siguiente manera.
Revisamos brevemente el algoritmo de aumento de gradiente y el modelo de
Poisson compuesto de Tweedie en la Sección 2 y la Sección 3,
respectivamente. Presentamos el desarrollo metodológico principal con
detalles de implementación en la Sección 4. En la Sección 5, Utilizamos
la simulación para mostrar la alta precisión predictiva de TDboost. Como
una aplicación, aplicamos

4

 TDboost para analizar los datos de un reclamo de seguro de automóvil
en la Sección 6.

2

Mejora de gradiente

El aumento de gradiente (Friedman, 2001) es un algoritmo de aprendizaje
automático no paramétrico recursivo que se ha utilizado con éxito en
muchas áreas. Muestra una notable flexibilidad en la resolución de
diferentes funciones de pérdida. Al combinar un gran número de
aprendices base, puede manejar interacciones de orden superior
Descargado por {[}McGill University Library{]} a las 18:16 28 de junio
de 2016

y producir formas funcionales altamente complejas. Proporciona una alta
precisión de predicción y, a menudo, supera a muchos métodos
competitivos, como regresión / clasificación lineal, embolsado (Breiman,
1996), splines y CART (Breiman et al., 1984). Para mantener el papel
autocontenido, explicamos brevemente los procedimientos generales para
el aumento de gradiente. Sea x = (x1, \ldots{}, xp) un vector de columna
p-dimensional para las variables predictoras y y sea la variable de
respuesta unidimensional. El objetivo es estimar la función de
predicción óptima \textasciitilde{} F () que mapea x a y minimizando el
valor esperado de una función de pérdida (,) sobre la clase de función
F: \textasciitilde{} F () = arg minEy, x {[}(y, F (X)){]}, F () F

donde se supone que es diferenciable con respecto a F. Dados los datos
observados \{yi, xi\} n, donde i = 1 \textasciitilde{} xi = (xi1,
\ldots{}, xip), la estimación de F () se puede realizar minimizando el
Función de riesgo empírico 1 min F () F n norte

(yi, F (xi)). i = 1

\begin{enumerate}
\def\labelenumi{(\arabic{enumi})}
\item
\end{enumerate}

Para el aumento de gradiente, se supone que cada función candidata FF es
un conjunto de aprendices base M METRO

F (x) = F

{[}0{]}

\begin{itemize}
\tightlist
\item
  m = 1
\end{itemize}

{[}m{]} h (x; {[}m{]}),

\begin{enumerate}
\def\labelenumi{(\arabic{enumi})}
\setcounter{enumi}{1}
\item
\end{enumerate}

donde h (x; {[}m{]}) generalmente pertenece a una clase de algunas
funciones simples de x llamados aprendices base (por ejemplo, regresión
/ árbol de decisión) con el parámetro {[}m{]} (m = 1, 2,, M). F {[}0{]}
es un escalar constante y

5

 {[}m{]} es el coeficiente de expansión. Tenga en cuenta que, a
diferencia de la estructura habitual de un modelo aditivo, no hay
restricciones en el número de predictores que se incluirán en cada h ()
y, por lo tanto, las interacciones de alto orden pueden considerarse
fácilmente utilizando esta configuración. Se adopta un algoritmo
escalonado avanzado para aproximar el minimizador de (1), que construye
los componentes {[}m{]} h (x; {[}m{]}) (m = 1, 2, \ldots{}, M)
secuencialmente a través de un gradiente de descenso -como
\textasciitilde{} enfoque. En cada etapa de iteración m (m = 1, 2,
\ldots{}), suponga que la estimación actual para F () es \^{} \^{} \^{}
F {[}m-1{]} (). Para actualizar la estimación de F {[}m-1{]} () a F
{[}m{]} (), el aumento de gradiente se ajusta a un negativo Descargado
por {[}McGill University Library{]} a las 18:16 28 de junio de 2016

vector de gradiente (como la respuesta de trabajo) a los predictores
utilizando un aprendiz base h (x; {[}m{]}). Este ajuste h (x; {[}m{]})
puede verse como una aproximación del gradiente negativo.
Posteriormente, el coeficiente de expansión {[}m{]} se puede determinar
mediante una minimización de búsqueda de línea con la función de riesgo
empírica \textasciitilde{}, y la estimación de F (x) para la siguiente
etapa se convierte en \^{} \^{} F {[}m{]} (x): = F {[} m-1{]} (x) +
{[}m{]} h (x; {[}m{]}), (3)

donde 0 \textless{}1 es el factor de contracción (Friedman, 2001) que
controla el tamaño del paso de actualización. Un pequeño impone más
contracción mientras que = 1 da pasos de gradiente negativo completos.
Friedman (2001) descubrió que el factor de contracción reduce el ajuste
excesivo y mejora la precisión predictiva.

3

Compuesto Poisson Distribución y Modelo Tweedie

En los problemas de predicción de primas de seguros, el monto total de
la reclamación para un riesgo cubierto generalmente tiene una
distribución continua en valores positivos, excepto por la posibilidad
de ser cero exacto cuando no se produce la reclamación. Un enfoque
estándar en la ciencia actuarial en el modelado de estos datos es el uso
de los modelos compuestos de Poisson de Tweedie, que presentamos
brevemente en esta sección. \textasciitilde{} Sea N una variable
aleatoria de Poisson indicada por Pois (), y las de Zd (d = 0, 1,
\ldots{}, N) sean variables aleatorias iid gamma (,) con media y
varianza 2. Supongamos que N es

6

\begin{description}
\tightlist
\item[]
independiente de Zd 's. Defina una variable aleatoria Z por 0 si N = 0.
Z = \textasciitilde{} Z + Z + + Z \textasciitilde{} 2 \textasciitilde{}
N si N = 1, 2,. . . 1
\end{description}

\begin{enumerate}
\def\labelenumi{(\arabic{enumi})}
\setcounter{enumi}{3}
\item
\end{enumerate}

Por lo tanto, Z es la suma de Poisson de variables aleatorias gamma
independientes. En las solicitudes de seguro, \textasciitilde{} se puede
ver Z como el monto total de la reclamación, N como el número de
reclamaciones reportadas y Zd como el

Pago del seguro por el reclamo dth. La distribución resultante de Z se
conoce como el compuesto Descargado por {[}McGill University Library{]}
a las 18:16 28 de junio de 2016

Distribución de Poisson (Jørgensen y de Souza, 1994; Smyth y Jørgensen,
2002), que se sabe que está estrechamente relacionada con los modelos de
dispersión exponencial (EDM) (Jørgensen, 1987). Tenga en cuenta que la
distribución de Z tiene una masa de probabilidad en cero: Pr (Z = 0) =
exp (-). Luego, basándose en que Z condicional en N = j es Gamma (j,),
la función de distribución de Z se puede escribir como j = 1

fZ (z \textbar{},,) = Pr (N = 0) d0 (z) + = exp (-) d0 (z) + j = 1

Pr (N = j) fZ \textbar{} N = j (z)

j e- z j-1 ez /, j! j (j)

donde d0 es la función delta de Dirac en cero y fZ \textbar{} N = j es
la densidad condicional de Z dada N = j. Smyth (1996) señaló que la
distribución de Poisson compuesta pertenece a una clase especial de EDM
conocidos como modelos Tweedie (Tweedie, 1984), que se definen por la
forma fZ (z \textbar{},) = a (z,) exp z - ( ), (5)

donde a () es una función normalizadora, () se denomina función
acumulativa, y tanto a () como () son conocidos. El parámetro está en R
y el parámetro de dispersión está en R +. Para los modelos de Tweedie,
la media E (Z) = () y la varianza Var (Z) = ¨ (), donde () y () son las
primeras y segundas derivadas de (), respectivamente. Los modelos de
Tweedie tienen la relación de poder media-varianza.

7

 Var (Z) = para algún parámetro de índice. Dicha relación
media-varianza da 1- 2-,, 1- 2- 1 2 =, () =. log, = 1 log, = 2 De hecho,
si reparametrize (,,) por = Descargado por {[}McGill University
Library{]} a las 18:16 28 de junio de 2016

\begin{enumerate}
\def\labelenumi{(\arabic{enumi})}
\setcounter{enumi}{5}
\item
\end{enumerate}

Se puede demostrar que la distribución compuesta de Poisson pertenece a
la clase de modelos Tweedie.

1 2-, 2-

=

2-, -1

= (- 1) -1,

\begin{enumerate}
\def\labelenumi{(\arabic{enumi})}
\setcounter{enumi}{6}
\item
\end{enumerate}

el modelo compuesto de Poisson tendrá la forma de un modelo Tweedie con
1 \textless{}\textless{}2 y\textgreater{} 0. Como resultado, para el
resto de este documento, solo consideramos el modelo (4), y simplemente
nos referimos a (4) como Modelo de Tweedie (o modelo de Poisson de
Tweedie compuesto), denotado por Tw (,,), donde 1
\textless{}\textless{}2 y\textgreater{} 0. Es sencillo mostrar que la
probabilidad de registro del modelo de Tweedie es log fZ (z
\textbar{},,) = 2- 1 1- - + log a (z,,), z 1- 2- (8)

donde la función de normalización a () se puede escribir como 1 zt zt =
1 Wt (z,,) = 1 (-1) tt (1+) (2-) tt! (t) t = 1 za (z,, ) = 1 y = (2 -) /
(- 1) y t = 1

para z\textgreater{} 0 para z = 0

,

Wt es un ejemplo de la función Bessel generalizada de Wright (Tweedie,

1984).

4

Nuestra propuesta

En esta sección, proponemos integrar el modelo Tweedie al algoritmo de
aumento de gradiente basado en árboles para predecir el tamaño de las
reclamaciones de seguros. Específicamente, nuestra discusión se enfoca
en modelar el seguro de automóvil personal como un ejemplo ilustrativo
(ver la Sección 6 para un análisis de datos reales), ya que nuestra
estrategia de modelado se extiende fácilmente a otras líneas de negocios
de seguros no de vida.

8

 Dada una póliza de seguro de automóvil i, sea Ni el número de
reclamaciones (conocido como la frecuencia de reclamación)
\textasciitilde{} y Zdi sea el tamaño de cada reclamación observada para
di = 1,. . . , Ni. Dejemos que sea la duración de la política, es decir,
el período de tiempo que la política permanece en vigor. Entonces Zi =
Ni di = 1

\textasciitilde{} Zdi es el monto total de la reclamación.

A continuación, nos interesa modelar la proporción entre la reclamación
total y la duración Yi = Zi / wi, una cantidad clave conocida como la
prima pura (Ohlsson y Johansson, 2010). Siguiendo la configuración del
modelo de Poisson compuesto, asumimos que Ni está distribuido en
Poisson, y su media i wi tiene una relación multiplicativa con la
duración wi, donde i es una política específica Descargado por {[}McGill
University Library{]} a las 18:16 28 de junio de 2016

parámetro que representa la frecuencia de reclamación esperada en la
duración de la unidad. Condicional a Ni, supongamos que las de Zdi (di =
1, \ldots{}, Ni) son iid Gamma (, i), donde i es un parámetro específico
de la política que determina la gravedad de la reclamación, y es una
constante. Además, asumimos que bajo la duración de la unidad (es decir,
wi = 1), la relación media-varianza de una política satisface Var (Yi) =
{[}E (Yi){]} para todas las políticas, donde Yi es la prima pura en la
duración de la unidad, es una constante, y = (+ 2) / (+ 1). Entonces, se
sabe que Yi Tw (i, / wi,), cuyos detalles se proporcionan en el Apéndice
Parte A. Luego, consideramos una cartera de pólizas \{(yi, xi, wi)\} n
de n contratos de seguros independientes , i = 1 donde para el contrato
i, yi es la prima pura de la póliza, xi es un vector de variables
explicativas que caracterizan al asegurado y el riesgo que se está
asegurando (por ejemplo, casa, vehículo), y wi es la duración.
Supongamos que la prima pura esperada i está determinada por una función
de predicción F: R p R de xi: log \{i\} = log \{E (Yi \textbar{} xi)\} =
F (xi). (9)

En este documento, no imponemos una restricción lineal u otra forma
paramétrica en F (). Dada la flexibilidad de F (), llamamos ajustes como
el modelo Tweedie mejorado (en oposición al GLM de Tweedie). Dado \{(yi,
xi, wi)\} n, la función log-verosimilitud se puede escribir como i = 1
norte

(F (),, \textbar{} \{yi, xi, wi\} n) i = 1

= i = 1 n

log fY (yi \textbar{} i, / wi,), 1- 2- wi i yi - i + log a (yi, / wi,).
1- 2-

= i = 1

\begin{enumerate}
\def\labelenumi{(\arabic{enumi})}
\setcounter{enumi}{9}
\item
\end{enumerate}

9

 4.1 Estimación de F () a través de TDboost Estimamos la función de
predicción F () integrando el modelo Tweedie reforzado en el algoritmo
de aumento de gradiente basado en niveles. Para desarrollar la idea,
asumimos eso y estamos dados por el momento. La estimación conjunta de F
(), y será estudiada en la Sección 4.2. Dado y, reemplazamos la función
objetivo general en (1) por la probabilidad de registro negativa
derivada en (10), y seleccionamos la función minimizadora F () sobre una
clase F de funciones básicas de aprendiz en la forma de (2). Es decir,
pretendemos estimar. Descargado por {[}McGill University Library{]} a
las 18:16 28 de junio de 2016 norte

F (x) = argmin - FF

(F (),, \textbar{} \{yi, xi, wi\} n) i = 1

= argmin FF i = 1

(yi, F (xi) \textbar{}),

\begin{enumerate}
\def\labelenumi{(\arabic{enumi})}
\setcounter{enumi}{10}
\item
\end{enumerate}

donde (yi, F (xi) \textbar{}) = wi - yi exp {[}(1 -) F (xi){]} exp {[}(2
-) F (xi){]} +. 1- 2-

Tenga en cuenta que, a diferencia de (11), la clase de función
seleccionada por Tweedie GLM (Smyth, 1996) está restringida a una
colección de funciones lineales de x. Proponemos aplicar el algoritmo
escalonado avanzado descrito en la Sección 2 para resolver (11). La
estimación inicial de F () se elige como una función constante que
minimiza la probabilidad negativa de loglik: norte

\^{} F {[}0{]} = argmin

= log

i = 1 ni = 1 wi yi ni = 1 wi

(yi, \textbar{}).

\^{} Esto corresponde a la mejor estimación de F sin covariables. Sea F
{[}m-1{]} la estimación actual antes de la iteración mth. En el paso
mth, ajustamos un aprendiz base h (x; {[}m{]}) a través de {[}m{]} n

= argmin {[}m{]} i = 1

{[}u {[}m{]} - h (xi; {[}m{]}){]} 2, i

\begin{enumerate}
\def\labelenumi{(\arabic{enumi})}
\setcounter{enumi}{11}
\item
\end{enumerate}

10

 donde (u {[}m{]}, \ldots{}, u {[}m{]}) es el gradiente negativo actual
de (\textbar{}), es decir, n 1 u {[}m{]} = - i (yi, F (xi) \textbar{}) F
(xi) (13) \^{} exp {[}(2 -) F {[}m-1{]} (xi){]}, (14)

= wi - yi exp {[}(1 -

\^{} F (xi) = F {[}m-1{]} (xi) \^{} {[}m-1{]} (xi){]} +) F

y usar un árbol de regresión del nodo L-terminal L

h (x; {[}m{]}) = l = 1

u {[}m{]} I (x R {[}m{]}) ll

\begin{enumerate}
\def\labelenumi{(\arabic{enumi})}
\setcounter{enumi}{14}
\item
\end{enumerate}

Descargado por {[}McGill University Library{]} a las 18:16 28 de junio
de 2016

L con parámetros {[}m{]} = \{R {[}m{]}, u {[}m{]}\} l = 1 como aprendiz
base. Para encontrar R {[}m{]} yu {[}m{]}, usamos un topl lll rápido

algoritmo de ``ajuste óptimo'' con un criterio de división de mínimos
cuadrados (Friedman et al., 2000) para encontrar las variables de
división y las ubicaciones de división correspondientes que determinan
las regiones terminales ajustadas L \{R {[}m{]}\} l = 1. Tenga en cuenta
que estimar la R {[}m{]} implica estimar la u {[}m{]} como la media que
cae en cada región: lll

u {[}m{]} = meani: xi R {[}m{]} (u {[}m{]}) li l

l = 1,. . . , L.

Una vez que se ha estimado el aprendiz base h (x; {[}m{]}), el valor
óptimo del coeficiente de expansión {[}m{]} se determina mediante una
búsqueda por línea norte

{[}metro{]}

= argmin i = 1 n

\^{} (yi, F {[}m-1{]} (xi) + h (xi;) \textbar{}) L

{[}metro{]}

(dieciséis)

= argmin i = 1

\^{} (yi, F {[}m-1{]} (xi) + l = 1

u {[}m{]} I (xi R {[}m{]}) \textbar{} ). ll

El árbol de regresión (15) predice un valor constante u {[}m{]} dentro
de cada región R {[}m{]}, por lo que podemos resolver (16) ll mediante
una búsqueda de línea separada realizada dentro de cada región
respectiva R {[}m{]}. El problema (16) reduce l para encontrar la mejor
constante {[}m{]} para mejorar la estimación actual en cada región R
{[}m{]} según el siguiente criterio: {[}m{]} = argmin \^{} l i: xi R
{[}m{]} l

\^{} (yi, F {[}m-1{]} (xi) + \textbar{}),

l = 1,. . . L

\begin{enumerate}
\def\labelenumi{(\arabic{enumi})}
\setcounter{enumi}{16}
\item
\end{enumerate}

11

 donde la solución está dada por {[}m{]} \^{} l i: xi R {[}m{]} l

= log

\^{} wi yi exp {[}(1 -) F {[}m-1{]} (xi){]} \^{} wi exp {[}(2 -) F
{[}m-1{]} (xi){]}

,

l = 1,. . . , L.

\begin{enumerate}
\def\labelenumi{(\arabic{enumi})}
\setcounter{enumi}{17}
\item
\end{enumerate}

i: xi R {[}m{]} l

\^{} Habiendo encontrado los parámetros \{{[}m{]}\} l = 1, luego
actualizamos la estimación actual F {[}m-1{]} (x) en cada \^{} l L
región correspondiente \^{} \^{} \^{} l F {[}m{]} (x) = F {[} m-1{]} (x)
+ {[}m{]} I (x R {[}m{]}), l Descargado por {[}McGill University
Library{]} a las 18:16 28 de junio de 2016

l = 1,. . . L

\begin{enumerate}
\def\labelenumi{(\arabic{enumi})}
\setcounter{enumi}{18}
\item
\end{enumerate}

donde 0 \textless{}1 es el factor de contracción. A continuación
(Friedman, 2001), establecimos = 0.005 en nuestra implementación. Más
discusiones sobre la elección de los parámetros de ajuste están en la
Sección 4.4. En resumen, el algoritmo completo de TDboost se muestra en
el algoritmo 1. El paso de impulso es \^{} repetido M veces y reportamos
F {[}M{]} (x) como la estimación final.

4.2 Estimación (,) a través de la probabilidad del perfil Siguiendo a
Dunn y Smyth (2005), utilizamos la probabilidad de perfil para estimar
la dispersión y el parámetro de índice, que determinan conjuntamente la
relación de varianza de media Var (Yi) = / wi de la prima pura.
Explotamos el hecho de que en los modelos de Tweedie la estimación de
depende solo de: dado un fijo, la estimación media () se puede resolver
en (11) sin saberlo. Luego, condicional a esto y al correspondiente (),
maximizamos la función de probabilidad de registro con respecto a () =
argmax ((),,),

\begin{enumerate}
\def\labelenumi{(\arabic{enumi})}
\setcounter{enumi}{19}
\item
\end{enumerate}

que es un problema de optimización univariable que se puede resolver
utilizando una combinación de búsqueda de sección dorada e interpolación
parabólica sucesiva (Brent, 2013). De tal manera, hemos determinado el
correspondiente ((), ()) para cada arreglo. Luego, adquirimos la
estimación maximizando la probabilidad del perfil con respecto a 50
valores espaciados igualmente \{1,. . . , 50\} en (0, 1): = argmax \{1,
\ldots{}, 50\}

((), (),).

\begin{enumerate}
\def\labelenumi{(\arabic{enumi})}
\setcounter{enumi}{20}
\item
\end{enumerate}

12



Algoritmo 1 TDboost \^{} 1. Inicializar F {[}0{]} Descargado por
{[}McGill University Library{]} a las 18:16 28 de junio de 2016

\^{} F {[}0{]} = log

ni = 1 wi yi ni = 1 wi

.

\begin{enumerate}
\def\labelenumi{\arabic{enumi}.}
\setcounter{enumi}{1}
\tightlist
\item
  Para m = 1,. . . , M repetidamente realiza los pasos 2. (a) 2. (d) 2.
  (a) Calcula el gradiente negativo (u {[}m{]}, \ldots{}, u {[}m{]}) n 1
  2. (b) Ajusta el gradiente negativo vector (u {[}m{]}, \ldots{}, u
  {[}m{]}) a (x1, \ldots{}, xn) mediante un árbol de regresión del nodo
  L-terminal n 1, donde xi = (xi1, \ldots{}, xip) para i = 1,. . . , n,
  dándonos las particiones L \{R {[}m{]}\} l = 1. l 2. (c) Calcule las
  predicciones óptimas del nodo terminal {[}m{]} para cada región R
  {[}m{]}, l = 1, 2,. . . , L ll {[}m{]} \^{} l = log i: xi R {[}m{]} l
\end{enumerate}

\^{} \^{} u {[}m{]} = wi - yi exp {[}(1 -) F {[}m-1{]} (xi){]} + exp
{[}(2 -) F {[}m-1{]} (xi){]} i

i = 1,. . . , n.

\^{} wi yi exp {[}(1 -) F {[}m-1{]} (xi){]} \^{} wi exp {[}(2 -) F
{[}m-1{]} (xi){]}

.

i: xi R {[}m{]} l

\^{} 2. (d) Actualice F {[}m{]} (x) para cada región R {[}m{]}, l = 1,
2,. . . , L l \^{} \^{} \^{} l F {[}m{]} (x) = F {[}m-1{]} (x) + {[}m{]}
I (x R {[}m{]}) l \^{} 3. Reporte F {[}M{]} (x) como estimación final l
= 1, 2,. . . , L.

13

 Finalmente, aplicamos en (11) y (20) para obtener las estimaciones
correspondientes () y (). Algunos problemas computacionales adicionales
para evaluar las funciones de probabilidad de registro en (20) y (21) se
discuten en el Apéndice Parte B.

4.3 interpretación del modelo Comparado con otros métodos de aprendizaje
estadístico no paramétrico, como las redes neuronales y las máquinas del
núcleo, nuestro nuevo estimador proporciona resultados interpretables.
En esta sección, discutimos algunas formas. Descargado por {[}McGill
University Library{]} a las 18:16 28 de junio de 2016

para la interpretación del modelo después de ajustar el modelo Tweedie
reforzado. 4.3.1 Efectos marginales de los predictores.

Los efectos principales y los efectos de interacción de las variables en
el modelo Tweedie mejorado pueden extraerse fácilmente. En nuestra
estimación, podemos controlar el orden de las interacciones
seleccionando el tamaño de árbol L (el número de nodos terminales) y el
número p de predictores. Un árbol con nodos terminales L produce una
función de aproximación de los predictores p con un orden de interacción
a lo sumo (L - 1, p). Por ejemplo, un tocón (L = 2) produce un modelo
TDboost aditivo con solo los efectos principales de los predictores, ya
que es una función basada en una única variable de división en cada
árbol. El ajuste L = 3 permite tanto efectos principales como
interacciones de segundo orden. Siguiendo a Friedman (2001) utilizamos
los llamados gráficos de dependencia parcial para visualizar los efectos
principales y los efectos de interacción. Dados los datos de
entrenamiento \{yi, xi\} n, con un vector de entrada p-dimensional i = 1
x = (x1, x2, . . . , xp), sea zs un subconjunto de tamaño s, tal que zs
= \{z1,. . . , zs\} \{x1,. . . , xp\}. Por ejemplo, para estudiar el
efecto principal de la variable j, establecemos el subconjunto zs =
\{zj\}, y para estudiar la interacción de segundo orden de las variables
i y j, establecemos zs = \{zi, zj\}. Sea z ~s el conjunto de
complementos \^{} de zs, de modo que z ~szs = \{x1,. . . , xp\}. Deje
que la predicción F (zs \textbar{} z ~s) sea una función del subconjunto
zs \^{} condicionado a valores específicos de z ~s. La dependencia
parcial de F (x) en zs puede entonces formularse tal que z ~szs = \{x1,.
. . , xp\}. Deje que la predicción F (zs \textbar{} z ~s) sea una
función del subconjunto zs \^{} condicionado a valores específicos de z
~s. La dependencia parcial de F (x) en zs puede entonces formularse tal
que z ~szs = \{x1,. . . , xp\}. Deje que la predicción F (zs \textbar{}
z ~s) sea una función del subconjunto zs \^{} condicionado a valores
específicos de z ~s. La dependencia parcial de F (x) en zs puede
entonces formularse

14

 \^{} como F (zs \textbar{} z ~s) promediado sobre la densidad marginal
del subconjunto del complemento z ~s \^{} F s (zs) = \^{} F (zs
\textbar{} z ~s) p ~s (z ~s) dz ~s, p (x) dz s es la densidad marginal
de z ~s. Estimamos (22) por (23) (22)

donde p ~s (z ~s) = 1 F s (zs) = n norte

i = 1

\^{} F (zs \textbar{} z ~s, i),

donde \{z ~s, i\} n son evaluados en los datos de entrenamiento. Luego
trazamos Fs (zs) contra zs. Tenemos ini = 1 Descargado por {[}McGill
University Library{]} a las 18:16 28 de junio de 2016

Se incluyó la función de diagrama de dependencia parcial en nuestro
paquete R ``TDboost''. Demostraremos esta funcionalidad en la Sección 6.
4.3.2 Importancia variable

En muchas aplicaciones es interesante identificar los factores
predictivos relevantes del modelo en el contexto de los métodos de
conjunto basados en árboles. El modelo TDboost define una medida de
importancia variable para cada predictor candidato X j en el conjunto X
= \{X1,. . . , X p\} en términos de predicción / explicación de la
respuesta Y. La principal ventaja de este procedimiento de selección de
variables, en comparación con los métodos de selección univariados, es
que el enfoque considera el impacto de cada predictor individual, así
como las interacciones multivariadas entre predictores simultáneamente .
Comenzamos por definir la medida de importancia variable (VI en
adelante) en el contexto de un árbol único. Presentado por primera vez
por Breiman et al. (1984), la VI medida IX j (T m) de la variable X j en
un solo árbol T m se define como la reducción de la heterogeneidad total
de la variable de respuesta Y producida \^{} por X j, que se puede
estimar sumando todas las disminuciones en las reducciones de error al
cuadrado que obtuve en todos los nodos internos L - 1 cuando se
selecciona X j como la variable de división. Indique v (X j) = l el
evento de que X j se seleccione como la variable de división en el nodo
interno l, y deje que I jl = I (v (X j) = l). Entonces L-1

IX j (T m) =

\^{} l I jl, l = 1

\begin{enumerate}
\def\labelenumi{(\arabic{enumi})}
\setcounter{enumi}{23}
\item
\end{enumerate}

15

 \^{} donde l se define como la diferencia de error al cuadrado entre
el ajuste constante y los dos ajustes de la sub-región (los ajustes de
la sub-región se logran dividiendo la región asociada con el nodo
interno l en las regiones izquierda y derecha). Friedman (2001) extendió
la VI medida IX j para el modelo de impulso con una combinación de
árboles de regresión M, promediando (24) sobre \{T 1,. . . , TM\}: 1 IX
j = M METRO

m = 1

IX j (Tm).

\begin{enumerate}
\def\labelenumi{(\arabic{enumi})}
\setcounter{enumi}{24}
\item
\end{enumerate}

A pesar del amplio uso de la medida VI, Breiman et al. (1984) y White y
Liu (1994) Descargado por {[}McGill University Library{]} a las 18:16 28
de junio de 2016

entre otros, han señalado que las medidas VI (24) y (25) están sesgadas:
incluso si X j es una variable no informativa de Y (no correlacionada
con Y), X j todavía puede seleccionarse como una variable de división,
por lo tanto la medida VI de X j no es cero por la ecuación (25).
Siguiendo a Sandri y Zuccolotto (2008) y Sandri y Zuccolotto (2010) para
evitar el sesgo de selección de variables, en este documento calculamos
una medida de VI ajustada para cada variable explicativa permutando cada
X j, los detalles computacionales se proporcionan en la Parte C del
Apéndice.

4.4 Implementación Hemos implementado nuestro método propuesto en un
paquete de R ``TDboost'', que está disponible públicamente en
Comprehensive R Archive Network en
\url{http://cran.r-project.org/web/packages/} TDboost / index.html.
Aquí, discutimos la elección de tres parámetros meta en el algoritmo 1:
L (el tamaño de los árboles), (el factor de contracción) y M (el número
de pasos de impulso). Para evitar un ajuste excesivo y mejorar las
predicciones fuera de la muestra, el procedimiento de refuerzo puede
regularizarse limitando el número de iteraciones de aumento M (detención
temprana; Zhang y Yu, 2005) y el factor de contracción. La evidencia
empírica (Friedman, 2001; Bè hlmann y Hothorn, 2007; Ridgeu way, 2007)
mostró que la precisión predictiva es casi siempre mejor con un factor
de contracción menor a costa de más tiempo de cómputo. Sin embargo, los
valores más pequeños de generalmente requieren un mayor número de
iteraciones de refuerzo M y, por lo tanto, inducen más tiempo de
computación (Friedman, 2001). Elegimos un ``suficientemente pequeño'' =
0.005 en todo y determinamos M por los datos.

dieciséis

 El valor L debe reflejar el verdadero orden de interacción en el
modelo subyacente, pero casi nunca tenemos ese conocimiento previo. Por
lo tanto, elegimos los valores óptimos de M y L utilizando la validación
cruzada de pliegues K, comenzando con un valor fijo de L. Los datos se
dividen en K pliegues de igual tamaño aproximadamente. Deje una función
de índice (i): \{1,. . . , n\} \{1,. . . , K\} indica el pliegue al que
se asigna la observación i. Cada vez, eliminamos el kth fold de los
datos (k = 1, 2, \ldots{}, K) y entrenamos el modelo utilizando los \^{}
{[}M{]} restantes K - 1 pliegues. Denotando por Fk (x) el modelo
resultante, calculamos la pérdida de validación al predecir cada késimo
vez de los datos eliminados: Descargado por {[}McGill University
Library{]} a las 18:16 28 de junio de 2016 norte

CV (M, L) =

1 n

i = 1

\^{} {[}M{]} (yi, F- (i) (xi; L) \textbar{}).

\begin{enumerate}
\def\labelenumi{(\arabic{enumi})}
\setcounter{enumi}{25}
\item
\end{enumerate}

Seleccionamos la M óptima en la que se alcanza la pérdida de validación
mínima ML = argmin CV (M, L). METRO

Si también tenemos que seleccionar L, repetimos todo el proceso para
varios L (por ejemplo, L = 2, 3, 4, 5) y seleccionamos el que tenga el
menor error de generalización mínimo L = argmin CV (L, ML). L

En un caso dado, el ajuste de árboles con una L más alta hace que se
requiera una M más pequeña para alcanzar el error mínimo.

5

Estudios de simulacion

En esta sección, comparamos TDboost con el modelo Tweedie GLM (TGLM:
Jørgensen y de Souza, 1994) y el modelo Tweedie GAM en términos del
rendimiento de estimación de la función. El modelo Tweedie GAM lo
propone Wood (2001), que se basa en un enfoque de spline de regresión
penalizado con selección automática de suavidad. Hay un paquete R
``MGCV'' que acompaña el trabajo, disponible en
\url{http://cran.r-project.org/web/packages/mgcv/index}. html. En todos
los ejemplos numéricos a continuación que utilizan el modelo TDboost, la
validación cruzada de cinco veces es

17

 adoptado para seleccionar el par óptimo (M, L), mientras que el factor
de contracción se establece en su valor predeterminado de 0.005.

5.1 Caso I En este estudio de simulación, demostramos que TDboost es
adecuado para ajustarse a funciones objetivo que no son lineales o que
involucran interacciones complejas. Consideramos dos funciones objetivo
reales: · Modelo 1 (función discontinua): la función objetivo es
discontinua según lo definido por F (x) = 0.5I (x\textgreater{} 0.5).
Suponemos que x Unif (0, 1) y y Tw (,,) con = 1.5 y = 0.5. · Modelo 2
(interacción compleja): la función objetivo tiene dos colinas y dos
valles. F (x1, x2) = e-5 (1-x1) + x2 + e-5x1 + (1-x2), que corresponde a
un escenario común en el que el efecto de una variable cambia según el
efecto de otra. Suponemos que x1, x2 Unif (0, 1) y y Tw (,,) con = 1.5 y
= 0.5. Generamos n = 1000 observaciones para entrenamiento y n = 1000
para pruebas, y ajustamos los datos de entrenamiento usando TDboost,
MGCV y TGLM. Dado que las verdaderas funciones de destino son conocidas,
norte 2 2 2 2

Descargado por {[}McGill University Library{]} a las 18:16 28 de junio
de 2016

i = 1

\^{} \textbar{} F (xi) - F (xi) \textbar{}

\^{} donde tanto la función de predicción real F (xi) como la función de
predicción F (xi) se evalúan en el conjunto de prueba. Los MAD
resultantes en los datos de prueba se informan en la Tabla 1, que tienen
un promedio de más de 100 repeticiones independientes. Las funciones
ajustadas del Modelo 2 se trazan en la Figura 2. En ambos casos,
encontramos que TDboost supera a TGLM y MGCV en términos de la capacidad
de recuperar las funciones verdaderas y da los errores de predicción más
pequeños.

18

 5.2 Caso II La idea es ver el rendimiento del estimador TDboost y el
estimador MGCV en una variedad de funciones de predicción muy
complicadas y generadas al azar, y estudiar cómo el tamaño del conjunto
de entrenamiento, la configuración de distribución y otras
características de los problemas afectan el rendimiento final de los dos
métodos. . Utilizamos el modelo ``generador de función aleatoria'' (RFG)
de Friedman (2001) en nuestra simulación. La verdadera función de
destino F se genera aleatoriamente como una expansión lineal de las
funciones \{gk\} 20: k = 1 Descargado por {[}McGill University
Library{]} a las 18:16 28 de junio de 2016 20

F (x) = k = 1

bk gk (zk).

\begin{enumerate}
\def\labelenumi{(\arabic{enumi})}
\setcounter{enumi}{26}
\item
\end{enumerate}

Aquí, cada coeficiente bk es una variable aleatoria uniforme de Unif
{[}-1, 1{]}. Cada gk (zk) es una función de zk, donde zk se define como
un subconjunto pk de la variable de diez dimensiones x en la forma zk =
\{xk (j)\} pk, j = 1 (28)

donde cada k es una permutación independiente de los enteros \{1,. . . ,
pag\}. El tamaño pk se selecciona aleatoriamente por min (2.5 + rk, p),
donde rk se genera a partir de una distribución exponencial con media 2.
Por lo tanto, el orden esperado de interacciones presentado en cada gk
(zk) es entre cuatro y cinco. Cada función gk (zk) es una función
gaussiana tridimensional: 1 gk (zk) = exp - (zk - uk) Vk (zk - uk), 2 Vk
se define por Vk = Uk Dk Uk, (30) (29 )

donde cada vector medio uk se genera aleatoriamente a partir de N (0, I
pk). La matriz de covarianza pk × pk

donde Uk es una matriz ortonormal aleatoria, Dk = diag \{dk {[}1{]},. .
. , dk {[}pk{]}\}, y la raíz cuadrada de cada elemento diagonal \{yi,
xi\} n de acuerdo con i = 1 yi Tw (i,,), xi N (0, I p), i = 1,. . . , n,
(31) dk {[}j{]} es una variable aleatoria uniforme de Unif {[}0.1,
2.0{]}. Generamos datos

donde i = exp \{F (xi)\}.

19

 Configuración I: cuando se conoce el índice En primer lugar,
estudiamos la situación de que el parámetro de índice verdadero se
conoce al ajustar modelos. Generamos datos según el modelo RFG con el
parámetro de índice = 1.5 y el parámetro de dispersión = 1 en el modelo
verdadero. Establecemos el número de predictores para que sean p = 10 y
generamos n \{1000, 2000, 5000\} observaciones como conjuntos de
entrenamiento, en los que tanto MGCV como TDboost están equipados con el
valor verdadero de 1.5. Se generó un conjunto de prueba adicional de n =
5000 observaciones para evaluar el desempeño de la estimación final.
Descargado por {[}McGill University Library{]} a las 18:16 28 de junio
de 2016

La Figura 3 muestra los resultados de la simulación para comparar el
rendimiento de estimación de MGCV y TDboost, al variar el tamaño de la
muestra de entrenamiento. Las distribuciones empíricas de las MAD que se
muestran como diagramas de caja se basan en 100 replicaciones
independientes. Podemos ver que en todos los casos, TDboost supera a
MGCV en términos de precisión de predicción. También probamos el
rendimiento de la estimación cuando el parámetro de índice está mal
especificado, es decir, usamos un valor de conjetura que difiere del
valor verdadero al ajustar el modelo TDboost. Debido a que es
estadísticamente ortogonal a y, lo que significa que los elementos fuera
de la diagonal de la matriz de información de Fisher son cero
(Jørgensen, 1997), esperamos que varíe muy lentamente a medida que
cambien. De hecho, utilizando \^{} \textasciitilde{} los datos de
simulación anteriores con el valor verdadero = 1.5 y = 1, ajustamos los
modelos TDboost con \textasciitilde{} nueve valores de suposición de
\{1.1, 1.2,. . . , 1.9\}. Los MAD resultantes se muestran en la Figura
5, que muestra que la elección del valor casi no tiene un efecto
significativo en la precisión de la estimación de. Configuración II: uso
del índice estimado A continuación, estudiamos la situación en la que se
desconoce el parámetro del índice verdadero, y usamos el estimado
obtenido del procedimiento de verosimilitud del perfil descrito en la
Sección 4.2 para ajustar el modelo. El mismo esquema de generación de
datos se adopta como en la Configuración I, excepto que ahora MGCV y
TDboost se ajustan estimando al máximo la probabilidad de perfil. La
Figura 6 muestra los resultados de la simulación para comparar el
rendimiento de estimación de MGCV y TDboost en dicha configuración.
Podemos ver que el uso del índice estimado A continuación, estudiamos la
situación en la que se desconoce el parámetro del índice verdadero, y
utilizamos el estimado obtenido del procedimiento de verosimilitud del
perfil descrito en la Sección 4.2 para ajustar el modelo. El mismo
esquema de generación de datos se adopta como en la Configuración I,
excepto que ahora MGCV y TDboost se ajustan estimando al máximo la
probabilidad de perfil. La Figura 6 muestra los resultados de la
simulación para comparar el rendimiento de estimación de MGCV y TDboost
en dicha configuración. Podemos ver que el uso del índice estimado A
continuación, estudiamos la situación en la que se desconoce el
parámetro del índice verdadero, y utilizamos el estimado obtenido del
procedimiento de verosimilitud del perfil descrito en la Sección 4.2
para ajustar el modelo. El mismo esquema de generación de datos se
adopta como en la Configuración I, excepto que ahora MGCV y TDboost se
ajustan estimando al máximo la probabilidad de perfil. La Figura 6
muestra los resultados de la simulación para comparar el rendimiento de
estimación de MGCV y TDboost en dicha configuración. Podemos ver que el
La Figura 6 muestra los resultados de la simulación para comparar el
rendimiento de estimación de MGCV y TDboost en dicha configuración.
Podemos ver que el La Figura 6 muestra los resultados de la simulación
para comparar el rendimiento de estimación de MGCV y TDboost en dicha
configuración. Podemos ver que el

20

 los resultados no tienen una diferencia significativa con los
resultados de la Configuración I: TDboost aún supera a MGCV en términos
de precisión de predicción cuando se utiliza el valor estimado en lugar
del verdadero. Por último, demostramos nuestros resultados de la
estimación de la dispersión y el índice utilizando la probabilidad del
perfil. Un total de 200 conjuntos de muestras de entrenamiento se
generan aleatoriamente a partir de un modelo verdadero de acuerdo con la
configuración (31) con = 2 y = 1.7, y cada muestra tiene 2000
observaciones. Encajamos el modelo TDboost en cada muestra y calculamos
las estimaciones en cada uno de los 50 valores espaciados
equitativamente \{1,. . . , 50\} en (1, 2). El (j, (j)) correspondiente
al máximo Descargado por {[}McGill University Library{]} a las 18:16 28
de junio de 2016

La probabilidad de perfil es la estimación de (,). El proceso de
estimación se repite 200 veces. Los índices estimados tienen una media =
1.68 y el error estándar SE () = 0.026, por lo que el valor verdadero =
1.7 está dentro de ± SE (). Las dispersiones estimadas tienen media =
1.82 y error estándar SE () = 0.12. La Figura 4 muestra la función de
probabilidad de perfil para una sola ejecución.

6

Aplicación: Reclamaciones de automóviles

6.1 conjunto de datos Consideramos un conjunto de datos de reclamaciones
de seguros de automóviles según lo analizado en Yip y Yau (2005) y Zhang
y Yu (2005). El conjunto de datos contiene 10,296 registros de vehículos
de conductor, cada registro incluye el monto total de reclamación de un
conductor individual (zi) en los últimos cinco años (wi = 5) y 17
características xi = (xi, 1, \ldots{}, xi, 17) para El conductor y el
vehículo asegurado. Queremos predecir la prima pura esperada basada en
xi. La tabla 3 resume el conjunto de datos. Las estadísticas
descriptivas de los datos se proporcionan en la Parte D del Apéndice. El
histograma de los montos totales de las reclamaciones en la Figura 1
muestra que la distribución empírica de estos valores es altamente
sesgada. Encontramos que aproximadamente el 61.1\% de los asegurados no
tenían reclamaciones, y aproximadamente el 29.6\% de los asegurados
tenían una reclamación positiva de hasta 10,000 dólares. Tenga en cuenta
que sólo 9. El 3\% de los asegurados tenía un monto de reclamo alto por
encima de los 10.000 dólares, pero la suma del monto de su reclamo
supuso el 64\% del monto total. Un-

21

 Otra característica importante de los datos es que hay interacciones
entre las variables explicativas. Por ejemplo, en la Tabla 2 podemos ver
que el efecto marginal de la variable REVOCADA en el monto total de la
reclamación es mucho mayor para los asegurados que viven en el área
urbana que los que viven en el área rural. La importancia de los efectos
de interacción se confirmará más adelante en nuestro análisis de datos.

6.2 modelos Separamos todo el conjunto de datos en un conjunto de
entrenamiento y un conjunto de pruebas con el mismo tamaño. Luego, el
TDDescargado por {[}McGill University Library{]} a las 18:16 28 de junio
de 2016

el modelo boost se ajusta al conjunto de entrenamiento y se ajusta con
una validación cruzada de cinco veces. Para comparación, también
ajustamos TGLM y MGCV, ambos se ajustan utilizando todas las variables
explicativas. En MGCV, las variables numéricas AGE, BLUEBOOK, HOMEKIDS,
KIDSDRIV, MVR PTS, NPOLICY, RETAINED y TRAVTIME se modelan mediante
términos suaves representados mediante el uso de splines de regresión
penalizados. Encontramos la suavidad adecuada para cada término de
modelo aplicable utilizando la Validación cruzada generalizada (GCV)
(Wahba, 1990). Para el modelo TDboost, no es necesario llevar a cabo la
transformación de datos, ya que el método de aumento basado en árbol
puede manejar automáticamente diferentes tipos de datos. Para otros
modelos, utilizamos la transformación logarítmica en BLUEBOOK, es decir,
log (BLUEBOOK), y escalamos todas las variables numéricas excepto
HOMEKIDS, KIDSDRIV, MVR PTS y NPOLICY tienen media 0 y desviación
estándar 1. También creamos variables ficticias para las variables
categóricas con más de dos niveles (CAR TYPE, JOBCLASS y MAX EDUC). Para
todos los modelos, utilizamos el método de probabilidad de perfil para
estimar la dispersión y el índice, que a su vez se utilizan para ajustar
los modelos finales.

6.3 Comparación de rendimiento Para examinar el rendimiento de TGLM,
MGCV y TDboost, después de ajustar el conjunto de entrenamiento,
predecimos el valor puro P (x) = (x) mediante la aplicación de cada
modelo en el conjunto de pruebas independientes. Sin embargo, se debe
prestar atención al medir las diferencias entre las primas previstas.

22

 P (x) y pérdidas reales y en los datos de prueba. La pérdida
cuadrática media o la pérdida absoluta media no es apropiada aquí porque
las pérdidas tienen altas proporciones de ceros y están muy sesgadas.
Por lo tanto, una estadística alternativa mide la curva de Lorenz
ordenada y el índice de Gini asociado propuesto por Frees et al. (2011)
se utilizan para capturar la discrepancia entre las distribuciones de
primas y pérdidas. Al calcular el índice de Gini, se puede comparar el
rendimiento de diferentes modelos predictivos. Aquí solo explicamos
brevemente la idea de la curva de Lorenz ordenada (Frees et al., 2011,
2013). Sea B (x) la ``prima base'', que se calcula utilizando la
descarga previa existente descargada por {[}McGill University Library{]}
a las 18:16 28 de junio de 2016

mi modelo de predicción, y deje que P (x) sea la ``prima competitiva''
calculada utilizando un modelo de predicción de prima alternativa. En la
curva de Lorenz ordenada, la distribución de las pérdidas y la
distribución de las primas se clasifican según la prima relativa R (x) =
P (x) / B (x). La distribución premium ordenada es \^{} DP (s) = ni = 1

B (xi) I (R (xi) s), ni = 1 B (xi)

y la distribución de pérdida ordenada es \^{} DL (s) = ni = 1 yi I (R
(xi) ni = 1 yi

\begin{enumerate}
\def\labelenumi{\alph{enumi})}
\setcounter{enumi}{18}
\item
\end{enumerate}

.

Dos distribuciones empíricas se basan en el mismo orden de
clasificación, lo que hace posible comparar las distribuciones de primas
y pérdidas para el mismo grupo de asegurados. La curva de Lorenz
ordenada es \^{} \^{} la gráfica de (DP (s), DL (s)). Cuando el
porcentaje de pérdidas es igual al porcentaje de primas para el
asegurador, la curva da como resultado una línea de 45 grados, conocida
como ``la línea de igualdad''. El doble del área entre la curva de
Lorenz ordenada y la línea de igualdad mide la discrepancia entre las
distribuciones de prima y pérdida, y se define como el índice de Gini.
Las curvas debajo de la línea de igualdad indican que, dado el
conocimiento de la prima relativa, una aseguradora podría identificar
los contratos rentables, cuyas primas son mayores que las pérdidas. Por
lo tanto, un índice Gini más grande (por lo tanto, un área más grande
entre la línea de igualdad y la curva a continuación) implicaría un
modelo más favorable. Siguiendo a Frees et al. (2013), especificamos
sucesivamente la predicción de cada modelo como la prima base B (x) y
utilizamos las predicciones de los modelos restantes como la prima
competidora P (x)

23

 Para calcular los índices de Gini. El procedimiento completo de la
división de datos y el cálculo del índice de Gini se repite 20 veces, y
en la Tabla 4 se presenta una matriz de los índices de Gini promediados
y los errores estándar. Para seleccionar el ``mejor'' modelo, utilizamos
una estrategia ``minimax'' (Frees et al., 2013) para seleccionar el
modelo premium base que sea menos vulnerable a los modelos premium
competidores; es decir, seleccionamos el modelo que proporciona el más
pequeño de los índices máximos de Gini, tomados sobre las primas de la
competencia. Encontramos que el índice máximo de Gini es 15.528 cuando
se usa B (x) = TGLM (x) como prima base, \^{} 12.979 cuando B (x) = MGCV
(x), y 4.000 cuando B (x) = TDboost (x) . Por lo tanto, TDboost tiene el
\^{} \^{} Descargado por {[}McGill University Library{]} a las 18:16 28
de junio de 2016

el índice máximo de Gini más pequeño en 4.000, por lo tanto, es el menos
vulnerable a puntajes alternativos. La Figura 7 también muestra que
cuando se selecciona TGLM (o MGCV) como la prima base, el área entre la
línea de igualdad y la curva de Lorenz ordenada es mayor al elegir
TDboost como la prima competidora, lo que indica nuevamente que el
modelo TDboost representa la más favorable elección.

6.4 Interpretación de los resultados. A continuación, nos centramos en
el análisis utilizando el modelo TDboost. Existen varias variables
explicativas relacionadas significativamente con la prima pura. La
medida VI y el valor de referencia de cada variable explicativa se
muestran en la Figura 8. Encontramos que REVOKED, MVR PTS, AREA y
BLUEBOOK tienen puntuaciones altas en la medida VI (la línea vertical),
y sus puntuaciones superan las líneas de base correspondientes (la línea
horizontal longitud de línea), lo que indica que la importancia de esas
variables explicativas es real. También encontramos que las variables
EDAD, CLASE DE TRABAJO, TIPO DE COCHE, NPOLICÍA, EDUCACIÓN MÁXIMA,
CASADO, NIÑO Y CAR USO tienen puntajes de medida VI más grandes que la
línea de base, pero las escalas absolutas son mucho menores que las
cuatro variables mencionadas anteriormente. Por otro lado, aunque la
medida VI de, por ejemplo, TRAVTIME es bastante grande, no supera
significativamente la importancia de la línea de base. Ahora usamos las
parcelas de dependencia parcial para visualizar el modelo ajustado. La
Figura 9 muestra los efectos principales de cuatro variables
explicativas importantes sobre la prima pura. Vemos claramente que los
fuertes efectos no lineales existen en los predictores BLUEBOOK y MVR
PTS: para los asegurados cuyos

24

 los valores del vehículo están por debajo de 40 K, su prima pura está
asociada negativamente con el valor del vehículo; después de que el
valor del vehículo pasa 40 K, la curva premium pura alcanza una meseta;
Además, la prima pura se asocia positivamente con los puntos de registro
del vehículo de motor MVR PTS, pero la curva premium pura alcanza una
meseta cuando la MVR PTS excede de seis. Por otro lado, las parcelas de
dependencia parcial sugieren que un asegurado que vive en el área urbana
(AREA = ``URBAN'') o con la licencia de conducir revocada (REVOKED =
``YES'') generalmente tiene una prima pura relativamente alta. En
nuestro modelo, la elección basada en datos para el tamaño del árbol es
L = 7, lo que significa que nuestro modelo Descargado por {[}McGill
University Library{]} a las 18:16 28 de junio de 2016

Incluye interacciones de orden superior. En la Figura 10, visualizamos
los efectos de cuatro interacciones importantes de segundo orden
utilizando los gráficos de dependencia parcial conjunta. Estas cuatro
interacciones son AREA × MVR PTS, AREA × NPOLICY, AREA × REVOKED y AREA
× TRAVTIME. Estas cuatro interacciones involucran la variable ÁREA:
podemos ver que los efectos marginales de MVR PTS, NPOLICY, REVOKED y
TRAVTIME sobre la prima pura son mayores para los asegurados que viven
en el área urbana (AREA = ``URBAN'') que los que viven en El área rural
(AREA = ``RURAL''). Por ejemplo, una fuerte interacción AREA × MVR PTS
sugiere que para los asegurados que viven en el área rural, los puntos
de registro de vehículos motorizados tienen un efecto marginal positivo
más débil en la prima pura esperada que para los asegurados que viven en
el área urbana.

7

Conclusiones

La necesidad de factores de riesgo no lineales, así como las
interacciones de los factores de riesgo para modelar el tamaño de las
reclamaciones de seguros es bien reconocida por los profesionales
actuariales, pero las herramientas prácticas para estudiarlos son muy
limitadas. En este documento, que no se basa en el supuesto lineal ni en
una estructura de interacción preespecificada, se ha diseñado un método
flexible de aumento de gradiente basado en árboles para el modelo de
Tweedie. Implementamos el método propuesto en un paquete de R fácil de
usar ``TDboost'' que puede hacer predicciones precisas de primas de
seguro para conjuntos de datos complejos y sirve como una herramienta
conveniente para que los profesionales actuariales investiguen los
efectos no lineales y de interacción. En el contexto del seguro de auto
personal,

25

 usamos implícitamente la duración de la póliza como una medida de
volumen (o exposición), y demostramos el desempeño de predicción
favorable de TDboost para la prima pura. En el caso de que se utilicen
medidas de exposición distintas a la duración, lo cual es común en los
seguros comerciales, podemos extender el método TDboost al tamaño de
reclamación correspondiente simplemente reemplazando la duración con
cualquier medida de exposición elegida. TDboost también puede ser un
complemento importante del modelo GLM tradicional en la calificación de
seguros. Incluso en las circunstancias estrictas en que los reguladores
exigen que el modelo final tenga Descargado por {[}McGill University
Library{]} a las 18:16 28 de junio de 2016

En una estructura GLM, nuestro enfoque aún puede ser muy útil debido a
su capacidad para extraer información adicional como no-monotonicidad /
no-linealidad e interacción importante. En el Apéndice Parte E,
proporcionamos un análisis de datos reales adicionales para demostrar
que nuestro método puede proporcionar información sobre la estructura de
los términos de interacción. Después de integrar la información obtenida
sobre los términos de interacción en el modelo GLM original, podemos
mejorar mucho la precisión general de la predicción de la prima de
seguro mientras mantenemos una estructura de modelo GLM. Además, vale la
pena mencionar que las aplicaciones del método propuesto pueden ir más
allá de la predicción de la prima de seguro y ser de interés para los
investigadores en muchos otros campos, incluidos la ecología (Foster y
Bravington, 2013), la meteorología (Dunn, 2004) y la ciencia política.
(Lauderdale, 2012). Ver, por ejemplo, Dunn y Smyth (2005) y Qian et al.
(2015) para descripciones de las amplias aplicaciones de distribución de
Tweedie. El método propuesto y la herramienta de implementación permiten
a los investigadores en estos campos relacionados aventurarse fuera del
marco de modelado de Tweedie GLM, construir nuevos modelos flexibles
desde perspectivas no paramétricas y usar las herramientas de
interpretación del modelo demostradas en nuestro análisis de datos
reales para estudiar sus propios problemas de interés.

26

 Referencias Anstey, KJ, Wood, J., Lord, S. y Walker, JG (2005),
``Factores cognitivos, sensoriales y físicos que permiten conducir con
seguridad en adultos mayores'', Clinical Clinical Psychology, 25, 4565.
Breiman, L. ( 1996), ``Bagging predictors,'' Machine learning, 24,
123140. - (1998), ``Arcing classifier (con discusión y una réplica del
autor),'' The Annals of Statistics, 26, 801849. Descargado por {[}McGill
University Library{]} a las 18:16 28 de junio de 2016

\begin{itemize}
\tightlist
\item
  (1999), ``Juegos de predicción y algoritmos de arco'', Neural
  Computation, 11, 14931517. Breiman, L., Friedman, J., Olshen, R.,
  Stone, C., Steinberg, D. y Colla, P . (1984), ``CART: Clasificación y
  árboles de regresión'', Wadsworth. Brent, RP (2013), Algoritmos para
  minimización sin derivados, Courier Dover Publications. B¨ hlmann, P.
  y Hothorn, T. (2007), ``Aumentar los algoritmos: Regularización,
  predicción y ajuste de modelo u'', Statistical Science, 22, 477505.
  Dionne, G., Gouri´ roux, C. y Vanasse, C. (2001), ``Prueba de
  evidencia de selección adversa en el mercado de seguros de
  automóviles: un comentario,'' Journal of Political Economy, 109,
  444453. Dunn, PK (2004), ``La ocurrencia y la cantidad de
  precipitación se pueden modelar simultáneamente ,''International
  Journal of Climatology, 24, 12311239.
\end{itemize}

27

 Foster, SD y Bravington, MV (2013), ``Un modelo de PoissonGamma para
el análisis de datos continuos no negativos ecológicos,'' Estadísticas
ambientales y ecológicas, 20, 533552. Frees, EW, Meyers, G. y Cummings,
AD (2011 ), ``Resumiendo los puntajes de los seguros utilizando un
índice de Gini'', Journal of the American Statistical Association, 106.
Frees, EWJ, Meyers, G., y Cummings, AD (2013), ``Elaboración de
calificaciones de seguros y un índice de Gini'', Journal of Risk y
seguros. Descargado por {[}McGill University Library{]} a las 18:16 28
de junio de 2016

Los elementos del aprendizaje estadístico: minería de datos, inferencia
y predicción. Segunda Edición., Serie Springer en Estadística, Springer.
Hastie, TJ y Tibshirani, RJ (1990), Modelos aditivos generalizados,
vol.~43, CRC Press. Jørgensen, B. (1987), ``Exponential dispersion
models,'' Journal of the Royal Statistical Society. Serie B
(Metodológica), 127162.

28

 - (1997), La teoría de los modelos de dispersión, vol.~76, CRC Press.
Jørgensen, B. y de Souza, MC (1994), ``Ajustando el modelo de Poisson
compuesto de Tweedie a los datos de reclamaciones de seguros'',
Scandinavian Actuarial Journal, 1994, 6993. Lauderdale, BE (2012),
``Compound PoissonGamma modelos de regresión para los resultados en
dólares que son a veces cero,''Political Analysis, 20, 387399.
Mildenhall, SJ (1999)," Una relación sistemática entre el sesgo mínimo y
el lineal generalizado Descargado por {[}McGill University Library{]} a
las 18:16 28 de junio de 2016

Revista de estadística computacional y gráfica, preimpresión. Renshaw,
AE (1994), ``Modelado del proceso de reclamaciones en presencia de
covariables'', ASTIN Bulletin, 24, 265285. Ridgeway, G. (2007),
``Modelos de regresión aumentada generalizada,'' Manual del paquete R.
Sandri, M. y Zuccolotto, P. (2008), ``Un algoritmo de corrección de
sesgo para la medida de importancia variable de Gini en los árboles de
clasificación'', Journal of Computational and Graphical Statistics, 17.

29

 - (2010), ``Análisis y corrección del sesgo en la disminución total de
las medidas de impureza de nodos para algoritmos basados en bases,''
Estadísticas y computación, 20, 393407. Lluvias, VE y Shotick, JA
(1994), ``Los efectos de las características del hogar en demanda de
seguro: un análisis de tobit,''Journal of Risk and Insurance, 492502.
Smyth, G. y Jørgensen, B. (2002)," Ajustando el modelo Poisson compuesto
de Tweedie a los datos de reclamaciones de seguros: Modelo de
dispersión, ``Boletín ASTIN, 32, 143157. Descargado por {[}McGill
University Library{]} a las 18:16 28 de junio de 2016

Smyth, GK (1996), ``Análisis de regresión de datos cuantitativos con
ceros exactos'', en Actas del segundo taller de Australia en Japón sobre
modelos estocásticos en ingeniería, tecnología y gestión, Citeseer,
pp.~572580. Tweedie, M. (1984), " Un índice que distingue entre algunas
familias exponenciales importantes, ``en Estadísticas: Aplicaciones y
Nuevas Direcciones: Proc. Conferencia Internacional Golden Jubilee del
Instituto de Estadística de la India, pp.~579604. Van de Ven, W. y van
Praag, BM (1981),''Aversión al riesgo y deducibles en el seguro de salud
privado: aplicación de un modelo de tobit ajustado a los gastos de
atención de salud familiar" Salud, economía y economía de la salud,
12548. Wahba, G. (1990), Modelos Spline para datos de observación,
vol.~59, SIAM. White, AP y Liu, WZ (1994), ``Nota técnica:

30

 Zhang, T. y Yu, B. (2005), ``Impulso con paradas tempranas:
convergencia y consistencia,'' The Annals of Statistics, 15381579.
Zhang, W. (2011), ``cplm: algoritmos EM de Monte Carlo y métodos
bayesianos para encajando modelos lineales compuestos de Tweedie
Poisson,''R package, \url{http://cran.r-project.org/web/} packages /
cplm / index.html.

Descargado por {[}McGill University Library{]} a las 18:16 28 de junio
de 2016

31



Descargado por {[}McGill University Library{]} a las 18:16 28 de junio
de 2016

Figura 1: Histograma de los datos de reclamación de seguro de automóvil
analizados en Yip y Yau (2005). Muestra que hay 6290 registros de
pólizas con cero reclamaciones totales por año de póliza, mientras que
los 4006 registros de pólizas restantes tienen pérdidas positivas.

32

 (a) F verdadero (x1, x2) \^{} (b) TDboost F (x1, x2)

2.5

2.5

2.0

2.0

F (x1, x2) 1.5

\^{} F (x1, x2) 1.5

Descargado por {[}McGill University Library{]} a las 18:16 28 de junio
de 2016

1.0

1.0

0.8 0.6 0.4 0.4 0.2 0.0 0.0 0.2 0.2 0.6

0.8 0.6 0.4 0.4 0.2 0.0 0.0 0.2 0.2 0.6

x2

x1

x2

x1

\^{} (c) TGLM F (x1, x2)

\^{} (d) MGCV F (x1, x2)

1.35 1.30 1.25 \^{} F (x1, x2) 1.20 1.15 1.10

2.0

\^{} F (x1, x2)

1.5

1.0

0.8 0.6 0.4 0.4 0.2 0.0 0.0 0.2 0.2 0.6

0.8 0.6 0.4 0.4 0.2 0.0 0.0 0.2 0.2 0.6

x2

x1

x2

x1

Figura 2: Curvas ajustadas que recuperan la función objetivo definida en
el Modelo 2. La figura superior izquierda muestra la función objetivo
verdadera. Las figuras superior derecha, inferior izquierda e inferior
derecha muestran las predicciones sobre los datos de prueba de TDboost,
TGLM y MGCV, respectivamente.

33



Descargado por {[}McGill University Library{]} a las 18:16 28 de junio
de 2016

Figura 3: Resultados de la simulación para el Ajuste I: compare el
rendimiento de estimación de MGCV y TDboost al variar el tamaño de la
muestra de entrenamiento y el parámetro de dispersión en el modelo
verdadero. Los diagramas de caja muestran distribuciones empíricas de
las MAD basadas en 100 replicaciones independientes.

34



Descargado por {[}McGill University Library{]} a las 18:16 28 de junio
de 2016

Figura 4: La curva representa la función de probabilidad de perfil de
una sola ejecución. La línea de puntos muestra el valor verdadero = 1.7.
La línea continua muestra el valor estimado = 1.68 correspondiente a la
probabilidad máxima. La dispersión estimada asociada es = 1.89.

35



Descargado por {[}McGill University Library{]} a las 18:16 28 de junio
de 2016

Figura 5: Resultados de la simulación para el Ajuste I cuando el índice
está mal especificado: el rendimiento de estimación de TDboost cuando se
varía el valor del parámetro de índice \{1.1, 1.2,. . . , 1.9\}. En el
\textasciitilde{} modelo verdadero = 1.5 y = 1. Los diagramas de caja
muestran distribuciones empíricas de las MAD basadas en 200
\textasciitilde{} replicaciones independientes.

36



Descargado por {[}McGill University Library{]} a las 18:16 28 de junio
de 2016

Figura 6: Resultados de simulación para el Ajuste II: compare el
rendimiento de estimación de MGCV y TDboost cuando se varía el tamaño de
la muestra de entrenamiento y el parámetro de dispersión en el modelo
verdadero. Los diagramas de caja muestran distribuciones empíricas de
las MAD basadas en 100 replicaciones independientes.

37



Descargado por {[}McGill University Library{]} a las 18:16 28 de junio
de 2016

Figura 7: Las curvas de Lorenz ordenadas para los datos de reclamación
de seguro de automóvil.

38



Descargado por {[}McGill University Library{]} a las 18:16 28 de junio
de 2016

Figura 8: Las medidas de importancia variable y líneas de base de 17
variables explicativas para modelar la prima pura.

39



Descargado por {[}McGill University Library{]} a las 18:16 28 de junio
de 2016

Figura 9: Efectos marginales de las cuatro variables explicativas más
significativas sobre la prima pura.

40



Descargado por {[}McGill University Library{]} a las 18:16 28 de junio
de 2016

Figura 10: Cuatro interacciones fuertes en parejas.

41



Descargado por {[}McGill University Library{]} a las 18:16 28 de junio
de 2016

Tabla 1: Las MAD promediadas y los errores estándar correspondientes se
basan en 100 replicaciones independientes. TGLM MGCV Modelo TDboost
0.1102 (0.0006) 0.0752 (0.0016) 0.0595 (0.0021) 1 2 0.3516 (0.0009)
0.2511 (0.0004) 0.1034 (0.0008)

42

 Tabla 2: El monto total promedio de reclamaciones para diferentes
categorías de los asegurados. ÁREA Urbana Rural No 3150.57 904.70
REVOCADA Sí 14551.62 7624.36 11401.05 6719.66 Diferencia

Descargado por {[}McGill University Library{]} a las 18:16 28 de junio
de 2016

43



Descargado por {[}McGill University Library{]} a las 18:16 28 de junio
de 2016

Tabla 3: Variables explicativas en el conjunto de datos del historial de
reclamaciones. Tipo N significa variable numérica, Tipo C significa
variable categórica. ID 1 2 3 4 5 6 7 8 9 10 11 12 13 14 15 16 EDAD
variable EDAD BLUEBOOK HOMEKIDS KIDSDRIV MVR PTS NPOLICY RETENIDO
TRAVTIME AREA USO DE AUTOMÓVIL TIPO DE CARRO GÉNERO JOBCLASS MAX EDUC
MARRIED REVOKED Tipo NNNNNNNNNCCCC Descripción Tipo de conductor Número
de vehículos de conducir niños Puntos de registro de vehículos
motorizados Número de pólizas Número de años como cliente Distancia al
trabajo Área de trabajo / hogar: Rural, Uso de vehículos urbanos:
Comercial, Privado Tipo de vehículo: Panel Truck, Pickup, Sedan, Sports
Car, SUV, Van Género del conductor: F, M Desconocido, Cuello azul,
Administrativo, Médico, Fabricante de casa, Abogado, Gerente,
Profesional, Nivel de educación del estudiante: Preparatoria o inferior,
Licenciatura, Preparatoria, Maestría, PhD Casado o no: Sí, No Si la
licencia Revocado en los últimos 7 años: Sí, No

44



Descargado por {[}McGill University Library{]} a las 18:16 28 de junio
de 2016

Tabla 4: Los índices de Gini promediados y los errores estándar en el
ejemplo de datos de reclamaciones de seguros de automóviles basados en
20 divisiones aleatorias. Competente Premium Base Premium TGLM MGCV
TDboost TGLM MGCV TDboost 0 7.833 (0.338) 15.528 (0.509) 3.044 (0.610) 0
12.979 (0.473) 0 4.000 (0.364) 3.540 (0.415)

45 Ver estadísticas de publicación


\end{document}
