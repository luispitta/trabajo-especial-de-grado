\documentclass[]{article}
\usepackage{lmodern}
\usepackage{amssymb,amsmath}
\usepackage{ifxetex,ifluatex}
\usepackage{fixltx2e} % provides \textsubscript
\ifnum 0\ifxetex 1\fi\ifluatex 1\fi=0 % if pdftex
  \usepackage[T1]{fontenc}
  \usepackage[utf8]{inputenc}
\else % if luatex or xelatex
  \ifxetex
    \usepackage{mathspec}
  \else
    \usepackage{fontspec}
  \fi
  \defaultfontfeatures{Ligatures=TeX,Scale=MatchLowercase}
\fi
% use upquote if available, for straight quotes in verbatim environments
\IfFileExists{upquote.sty}{\usepackage{upquote}}{}
% use microtype if available
\IfFileExists{microtype.sty}{%
\usepackage{microtype}
\UseMicrotypeSet[protrusion]{basicmath} % disable protrusion for tt fonts
}{}
\usepackage[margin=1in]{geometry}
\usepackage{hyperref}
\hypersetup{unicode=true,
            pdfborder={0 0 0},
            breaklinks=true}
\urlstyle{same}  % don't use monospace font for urls
\usepackage{graphicx,grffile}
\makeatletter
\def\maxwidth{\ifdim\Gin@nat@width>\linewidth\linewidth\else\Gin@nat@width\fi}
\def\maxheight{\ifdim\Gin@nat@height>\textheight\textheight\else\Gin@nat@height\fi}
\makeatother
% Scale images if necessary, so that they will not overflow the page
% margins by default, and it is still possible to overwrite the defaults
% using explicit options in \includegraphics[width, height, ...]{}
\setkeys{Gin}{width=\maxwidth,height=\maxheight,keepaspectratio}
\IfFileExists{parskip.sty}{%
\usepackage{parskip}
}{% else
\setlength{\parindent}{0pt}
\setlength{\parskip}{6pt plus 2pt minus 1pt}
}
\setlength{\emergencystretch}{3em}  % prevent overfull lines
\providecommand{\tightlist}{%
  \setlength{\itemsep}{0pt}\setlength{\parskip}{0pt}}
\setcounter{secnumdepth}{0}
% Redefines (sub)paragraphs to behave more like sections
\ifx\paragraph\undefined\else
\let\oldparagraph\paragraph
\renewcommand{\paragraph}[1]{\oldparagraph{#1}\mbox{}}
\fi
\ifx\subparagraph\undefined\else
\let\oldsubparagraph\subparagraph
\renewcommand{\subparagraph}[1]{\oldsubparagraph{#1}\mbox{}}
\fi

%%% Use protect on footnotes to avoid problems with footnotes in titles
\let\rmarkdownfootnote\footnote%
\def\footnote{\protect\rmarkdownfootnote}

%%% Change title format to be more compact
\usepackage{titling}

% Create subtitle command for use in maketitle
\newcommand{\subtitle}[1]{
  \posttitle{
    \begin{center}\large#1\end{center}
    }
}

\setlength{\droptitle}{-2em}

  \title{}
    \pretitle{\vspace{\droptitle}}
  \posttitle{}
    \author{}
    \preauthor{}\postauthor{}
    \date{}
    \predate{}\postdate{}
  

\begin{document}

\%\%\%\%\%\%\%\%\%\%\%\%\%\%\%\%\%\%\%\%\%\%\%\%\%\%\%\%\%\%\%\%\%\%\%\%\%\%\%\%\%\%\%\%\%\%\%\%\%\%\%\%\%\%\%\%\%\%\%\%\%\%\%\%\%\%\%\%\%\%\%\%\%\%\%
\%\%\% Plantilla para elaborar tesis y tareas en Markdown \%\%\% Título:
Tesis para el Posgrado en Ciencias de la Administración, UNAM \%\%\%
Autor: Luis O. Ramírez Fernández \textless{}
\href{mailto:lramirez88mx@comunidad.unam.mx}{\nolinkurl{lramirez88mx@comunidad.unam.mx}}\textgreater{}
\%\%\% Fecha: 18 de marzo de 2016 \%\%\% La vesión más reciente en
\url{https://github.com/opengraphix/plantilla_tesis_md} \%\%\% Esta
plantilla esta inspirada en los siguientes trabajo para elaborar tesis.
\%\%\% - Jesús Velázquez \& Marco Ruiz. 2012. Tesis FI UNAM.
\url{https://github.com/Tepexic/Tesis-FI-UNAM} \%\%\% - Tom Pollard.
2015, febrero 2. Template for writing a PhD thesis in Markdown.
\url{https://github.com/tompollard/phd_thesis_markdown}
\%\%\%\%\%\%\%\%\%\%\%\%\%\%\%\%\%\%\%\%\%\%\%\%\%\%\%\%\%\%\%\%\%\%\%\%\%\%\%\%\%\%\%\%\%\%\%\%\%\%\%\%\%\%\%\%\%\%\%\%\%\%\%\%\%\%\%\%\%\%\%\%\%\%\%
\documentclass[letterpaper, $fontsize$, spanish, twoside]{report} %book
\usepackage{multirow,booktabs,setspace,caption, subcaption}

%Paquetes
% Typografia
%\usepackage{fontspec}
%\usepackage{charter} %% Charter
%\renewcommand{\familydefault}{bch}
%\usepackage{avant} %% Avant Grade
%\renewcommand{\familydefault}{pag}
%\usepackage{palatino} %% Palatino
%\renewcommand{\familydefault}{ppl}
%\usepackage{helvet} %% Helvetica
%\renewcommand{\familydefault}{phv}
%\renewcommand{\familydefault}{\sfdefault}
%\usepackage{ebgaramond}
%\usepackage[ttscale=0.85]{libertine} %% Fuente Libertine Linux
\usepackage[proportional]{libertine} %% Fuente Libertine Linux
\usepackage{lettrine}
\renewcommand{\ttfamily}{\fontencoding{OT1}\fontfamily{cmtt}\selectfont}
\usepackage[T1]{fontenc}

\usepackage{cite} %Referenciar citas en biblitex
\usepackage[natbibapa]{apacite}
\usepackage{hyperref}
%% Esto es para poder escribir acentos directamente:
\usepackage[utf8]{inputenc}
%\usepackage{enumerate}
%% Esto es para que LaTeX sepa que el texto está en español
\usepackage[spanish,activeacute, es-tabla]{babel}
\usepackage[spanish]{babelbib}
%% Citas
\usepackage{upquote}
\usepackage{microtype}
% Tablas
\usepackage{longtable}
\usepackage{grffile}

%% Parametros para margenes del documento
$if(geometry)$
\usepackage[$for(geometry)$$geometry$$sep$,$endfor$]{geometry}
$endif$
\setlength{\parskip}{9pt}
\setlength\parindent{0pt} % indentation
\renewcommand{\baselinestretch}{$linestretch$} %Espaciado de línea 1.5 puntos
\usepackage{siunitx}

\usepackage{setspace}
\onehalfspacing % Adjust spacing between lines to 1.5
% \doublespacing
\raggedbottom

% fix for pandoc 1.14
\providecommand{\tightlist}{%
  \setlength{\itemsep}{0pt}\setlength{\parskip}{0pt}}

% TP: hack to truncate list of figures/tables.
\usepackage{truncate}
%\usepackage{caption}
\usepackage{tocloft}
% TP: end hack

$if(linestretch)$
\usepackage{setspace}
\setstretch{$linestretch$}
$endif$

%% Table of contents formatting
%\renewcommand{\contentsname}{Índice}
\setcounter{tocdepth}{3}

%% Para matemáticas
\usepackage{amsmath, amssymb, mathtools}
\DeclareMathSizes{12}{13}{7}{7}

$if(numbersections)$
\setcounter{secnumdepth}{5}
$else$
\setcounter{secnumdepth}{4}
$endif$

%% Subtitulos
\usepackage{sectsty}
\usepackage[normalem]{ulem}
\sectionfont{\rmfamily\bfseries\large}
\subsectionfont{\rmfamily\bfseries\scshape\normalsize}
\subsubsectionfont{\rmfamily\bfseries\upshape\normalsize}

$if(verbatim-in-note)$
\VerbatimFootnotes % Notas de pie
$endif$
$for(header-includes)$
$header-includes$
$endfor$

% Listas y viñetas:
\providecommand{\tightlist}{%
  \setlength{\itemsep}{0pt}\setlength{\parskip}{0pt}}
$if(listings)$
\usepackage{listings}
 $endif$

% Viñetas enumeras:
$if(fancy-enums)$
\makeatletter\AtBeginDocument{%
  \renewcommand{\@listi}
    {\setlength{\labelwidth}{4em}}
}\makeatother
\usepackage{enumerate}
$endif$

$if(tables)$
\usepackage{ctable}
\usepackage{float} % Provee la opción H flotantes de los elementos
$endif$
% Sintaxis resaltada de cófigo
$if(highlighting-macros)$
  $highlighting-macros$
$endif$
$if(tables)$
\usepackage{array}
% This is needed because raggedright in table elements redefines \\:
\newcommand{\PreserveBackslash}[1]{\let\temp=\\#1\let\\=\temp}
\let\PBS=\PreserveBackslash
$endif$
$if(links)$
\usepackage[breaklinks=true]{hyperref}
$endif$
$if(url)$
\usepackage{url}
$endif$

%% Uso de imágenes y graficos
$if(graphics)$
\usepackage{graphicx}
\makeatletter
\def\maxwidth{\ifdim\Gin@nat@width>\linewidth\linewidth\else\Gin@nat@width\fi}
\def\maxheight{\ifdim\Gin@nat@height>\textheight\textheight\else\Gin@nat@height\fi}
\makeatother
% Scale images if necessary, so that they will not overflow the page
% margins by default, and it is still possible to overwrite the defaults
% using explicit options in \includegraphics[width, height, ...]{}
\setkeys{Gin}{width=\maxwidth,height=\maxheight,keepaspectratio}
$endif$

%%%------------------------------------------------------------------------
%%% Registro de versiones con Git
%%%------------------------------------------------------------------------

%% Si no quieres usar git o el paquete vc (desde CTAN), comenta la línea correspondientes.
%% Si comentas, debes estar seguro de comentar toda la sección
\immediate\write18{sh ./vc.sh}
\input{vc}

%%%------------------------------------------------------------------------
%%% Definiendo la configuración del encabezado
%%%------------------------------------------------------------------------
\usepackage{fancyhdr}
%\pagestyle{plain}

\pagestyle{fancy}
\renewcommand{\chaptermark}[1]{\markboth{\MakeUppercase{\thechapter. #1 }}{}}
\renewcommand{\sectionmark}[1]{\markright{\thesection\ #1}}

\fancyhf{}
\fancyhead[RO]{\sffamily\scshape\rightmark}
\fancyhead[LE]{\sffamily\scshape\leftmark}
\fancyfoot[C]{\thepage}
\renewcommand{\headrulewidth}{0.5pt}
\renewcommand{\footrulewidth}{0.1pt}

\addtolength{\headheight}{0.5pt}
\fancypagestyle{plain}{
  \fancyhead{}
  \renewcommand{\headrulewidth}{0pt}
}

%%%------------------------------------------------------------------------
%%% Definiento la numeración
%%%------------------------------------------------------------------------
{\newpage\renewcommand{\thepage}{\arabic{page}}\setcounter{page}{1}}

%%%------------------------------------------------------------------------
%%% Configuración de estilo de encabezados de capítulos
%%%------------------------------------------------------------------------
% Ver en: http://aristarco.com.es/recetario-latex/trucos-y-consejos/cambiar-titulos-partes-capitulos

\usepackage{titlesec}
\newcommand{\bigrule}{\titlerule[0.5mm]}
\titleformat{\chapter}[display] % CAMBIAMOS EL FORMATO DE LOS CAPÍTULOS
{\bfseries\Huge} % por defecto se usarán caracteres de tamaño \Huge en negrita
{% contenido de la etiqueta
\titlerule % línea horizontal
\filleft % texto alineado a la derecha
\Large\chaptertitlename\ % "Capítulo" o "Apéndice" en tamaño \Large en lugar de \Huge
\Large\thechapter} % número de capítulo en tamaño \Large
{0mm} % espacio mínimo entre etiqueta y cuerpo
{\filleft} % texto del cuerpo alineado a la derecha
[\vspace{0.5mm} \bigrule] % después del cuerpo, dejar espacio vertical y trazar línea horizontal gruesa

%%%------------------------------------------------------------------------
%%% Tablas
%%%------------------------------------------------------------------------
\usepackage{threeparttable}
\usepackage{array}
\newcolumntype{x}[1]{%
>{\centering\arraybackslash}m{#1}}%

%%%------------------------------------------------------------------------
%%% Tablas e imagenes estilo APA
%%% Ver en: https://tex.stackexchange.com/questions/254254/how-to-set-figures-and-tables-captions-in-apa-style-without-using-apa6-class
%%%------------------------------------------------------------------------
\DeclareCaptionLabelSeparator*{spaced}{\\[2ex]}
\captionsetup[table]{textfont=it,font=footnotesize,format=plain,justification=justified,
  singlelinecheck=false,labelsep=spaced,skip=0pt,labelfont=bf}
\captionsetup[figure]{labelsep=period,labelfont={it,bf},font={footnotesize,doublespacing},justification=justified,
  singlelinecheck=false}

%%%------------------------------------------------------------------------
%%% Imagenes en horizontal
%%%------------------------------------------------------------------------
\usepackage{lscape}
\newcommand{\blandscape}{\begin{landscape}}
\newcommand{\elandscape}{\end{landscape}}

%%%------------------------------------------------------------------------
%%% Cuerpo del Documento
%%%------------------------------------------------------------------------

\textbackslash{}begin\{document\}

\(if(alignment)\) \textbackslash{}begin\{\(alignment\)\} \(endif\)

\%\(for(include-before)\) \%\(include-before\)

\(endfor\) \(if(toc)\) \{ \hypersetup{linkcolor=black}
\setcounter{tocdepth}{$toc-depth$} \tableofcontents
 \} \(endif\) \(if(lot)\) \listoftables
 \(endif\) \(if(lof)\) \listoffigures
 \(endif\)

\(body\)

\(if(natbib)\) \(if(biblio-files)\) \(if(biblio-title)\)
\(if(book-class)\) \renewcommand\bibname{$biblio-title$} \(else\)
\renewcommand\refname{$biblio-title$} \(endif\) \(endif\)
\bibliography{$biblio-files$}

\(endif\) \(endif\) \(if(biblatex)\)
\printbibliography\(if(biblio-title)\){[}title=\(biblio-title\){]}\(endif\)

\(endif\) \(for(include-after)\) \(include-after\)

\(endfor\)

\textbackslash{}end\{document\}


\end{document}
